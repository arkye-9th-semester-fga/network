Qual é o papel do arquivo \texttt{/etc/hosts} no processo de resolução de  nomes?

\textbf{R}: O arquivo \texttt{/etc/hosts} armazena pares de endereço de \texttt{IP} (tanto para \text{IPv4} quanto para \text{IPv6}) e \textbf{nome do \emph{host}}, de forma que, ao inserir o \textbf{nome do \emph{host}}, este redirecione para o endereço de \texttt{IP} correspondente quando se tratar de uma transação que exija esse protocolo.

Por exemplo, ao definir os seguintes pares no arquivo \texttt{/etc/hosts}:
\begin{lstlisting}
	# IPv4
	127.0.0.1 localhost
	# IPv6
	::1 localhost
\end{lstlisting}

Sempre que realizar uma requisição utilizando o protocolo \texttt{IP} com o \textbf{nome de \emph{host}} \texttt{localhost} será redirecionado o valor \texttt{127.0.0.1} (\texttt{IPv4}) ou o valor \texttt{::1} (\texttt{IPv6}).
