Em vários dos protocolos ora estudados, foi presenciada uma etapa de autorização que preparava uma sessão para a recepção de comandos de determinado cliente. O \texttt{SMTP} demonstrou-se um protocolo que não demanda uma etapa de autorização. Em que momento isso acontece? O fato de essa etapa ser suprimida resulta em algum risco para um serviço de \emph{e-mail}?

\textbf{R}: Não há autenticação de emissor durante a definição do mesmo. Isto significa que, ao ter acesso interno em um servidor que possui interface de comunicação através do protocolo \texttt{SMTP}, é possível enviar mensagens de qualquer usuário identificado neste servidor para qualquer outro servidor \texttt{SMTP} válido. O \emph{spamming} é a principal consequência dessa limitação de segurança no protocolo \texttt{SMTP}.
